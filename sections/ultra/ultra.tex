\documentclass[../../rapport.tex]{subfiles}

\begin{document}
  Pour montrer l'unicité dans le théorème de Bowen, on a besoin d'un résultat de topologie de certains espaces,
  appelés espaces ultramétriques où l'inégalité triangulaire habituelle est rermplacée par une inégalité dite ultramétrique :
  $$\forall x, y, z \in E, d(x, z) \le \max{(d(x, y), d(y, z))}.$$
  Cette inégalité donne de nombreux résultats topologiques très différents des espaces métriques,
  notamment sur les triangles et les boules.

  Le théorème permettant de prouver l'unicité du théorème de Bowen est le suivant,
  et nous allons en donner une preuve dans cette section,
  puis une formalisation possible en Lean.

  \begin{theorem}
    Soit $(E, d)$ un espace ultramétrique et $\O$ un ouvert de $E$,
    alors il existe une partie $R \subseteq \O$ et une application $r \colon R \longrightarrow \R^*_+$ tels que
    $$\O = \bigsqcup_{x \in R}{B(x, r(x))}.$$
    De plus, si $E$ est séparable alors on peut imposer que $R$ soit au plus dénombrable.
  \end{theorem}

  On notera dans la suite $\O \subseteq E$ un ouvert et donc on a pour chaque $x \in \O$ un réel $r_x > 0$ tel que
  $B(x, r_x) \subseteq \O$. On a alors
  $$\O = \bigcup_{x\in\O}{B(x, r_x)}.$$

  \begin{remark}
    On peut voir les $(r_x)_{x\in\O}$ comme une une application $r \colon \mathcal{O} \longrightarrow \R^*_+$.
  \end{remark}

  \begin{definition}
    On note $B(\O)$ l'ensemble des boules incluses dans $\O$, en particulier :
    $$B(\O) := \left\{b | \exists x \in \O, \exists r > 0, B(x, r) \subseteq \O \right\}.$$
    On définit une relation sur les boules contenues dans $\O$ de la manière suivante :
    $$\forall u, v \subseteq \O$$ % TODO: A finir
  \end{definition}

\end{document}
