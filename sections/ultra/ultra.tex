\documentclass[../../rapport.tex]{subfiles}

\begin{document}
  Pour montrer l'unicité dans le théorème de Bowen, on a besoin d'un résultat de topologie de certains espaces,
  appelés espaces ultramétriques, qui sont des espaces métriques où l'inégalité triangulaire habituelle est remplacée
  par une inégalité dite ultramétrique :
  $$\forall x, y, z \in E, d(x, z) \le \max{(d(x, y), d(y, z))}.$$
  Cette inégalité donne de nombreux résultats topologiques très différents des espaces métriques,
  notamment sur les triangles et les boules.

  Le théorème permettant de prouver l'unicité du théorème de Bowen est le suivant,
  et nous allons en donner une preuve dans cette section,
  puis une formalisation possible en Lean.

  \begin{theorem}
    \label{thm:open_bdd_eq_disjoint_union_ball}
    Soit $(E, d)$ un espace ultramétrique et $\O$ un ouvert borné de $E$,
    alors il existe une partie $R \subseteq \O$ et une application $r \colon R \longrightarrow \R^*_+$ tels que
    $$\O = \bigsqcup_{x \in R}{B(x, r(x))}.$$
    % De plus, si $E$ est séparable alors on peut imposer que $R$ soit au plus dénombrable.
  \end{theorem}

  Dans la suite on fixe $(E, d)$ un espace ultramétrique
  Commençons par quelques lemmes décrivant les relations entres les boules dans ces espaces.

  \begin{lemma}
    \label{lem:ball_eq_of_mem}
    Soit $x, y \in E$ et un réel $r > 0$, si $y \in B(x, r)$, alors $B(x, r) = B(y, r)$
  \end{lemma}

  \begin{proof}
    Supposons que $y \in B(x, r)$. Soit $z \in B(x, r)$, alors
    $$d(y, z) \le \max{(d(y, x), d(x, z))} < \max{(r, r)} = r,$$
    car $y, z \in B(x, r)$. Donc on a une inclusion : $B(x, r) \subseteq B(y, r)$.

    Réciproquement comme $y \in B(x, r)$ on a $x \in B(y, r)$ par symétrie de la distance
    et donc avec le même raisonnement on a l'autre inclusion.
    Finalement, $B(x, r) = B(y, r)$.
  \end{proof}

  \begin{lemma}
    \label{lem:ball_subset_trichotomy}
    Soit $x, y \in E$ et deux réels $r, s > 0$, alors il y a trois possibilités :
    \begin{enumerate}
      \item $B(x, r) \subseteq B(y, s)$,
      \item $B(y, s) \subseteq B(x, r)$,
      \item $B(x, r) \cap B(y, s) = \emptyset$.
    \end{enumerate}
  \end{lemma}

  \begin{proof}
    Supposons que $B(x, r) \cap B(y, s) \not= \emptyset$, alors soit $z \in B(x, r) \cap B(y, s)$.
    On peut alors écrire $B(x, r) = B(z, r)$ et $B(y, s) = B(z, s)$ grâce au lemme \ref{lem:ball_eq_of_mem},
    et ainsi on a bien le résultat voulu en fonction de l'ordre de $r$ et $s$.
  \end{proof}

  \begin{lemma}
    \label{lem:union_mem_balls}
    Soit $(x_i)_{i\in I}$ une famille de point de $E$ et $(r_i)_{i\in I}$ une famille de réels strictement positifs.
    De plus, supposons qu'il existe $x \in E$ tel que $x \in B(x_i, r_i)$ pour tout $i \in I$ et que $R = \sup_{i\in I}{r_i} < \infty$.
    Alors
    $$\bigcup_{i \in I}B(x_i, r_i) = B(x, R).$$
  \end{lemma}

  \begin{proof}
    Commençons par réécrire chaque boule en changeant son centre grâce au lemme \ref{lem:ball_eq_of_mem} et
    notons $R = \sup_{i\in I}{r_i} < \infty$, on a alors l'inclusion
    $$\bigcup_{i\in I}B(x_i, r_i) = \bigcup_{i\in I}{B(z, r_i)} \subseteq B(z, R).$$

    Réciproquement si $y \in B(z, R)$, alors il existe $i_0 \in I$ tel que $d(y, z) < r_{i_0} \le R$
    et donc $y \in B(z, r_{i_0}) \subseteq \bigcup_{i\in I}{B(z, r_i)}$.
    Finalement, l'union de ces boules est encore une boule.
  \end{proof}

  On notera dans la suite $\O \subseteq E$ un ouvert borné,
  et donc on a pour chaque $x \in \O$ un réel $r_x > 0$ tel que $B(x, r_x) \subseteq \O$.
  On a alors
  $$\O = \bigcup_{x\in\O}{B(x, r_x)}.$$

  \begin{remark}
    On peut voir les $(r_x)_{x\in\O}$ comme une une application $r \colon \mathcal{O} \longrightarrow \R^*_+$.
  \end{remark}

  \begin{definition}
    On note $B(\O)$ l'ensemble des boules incluses dans $\O$, en particulier :
    $$B(\O) := \left\{B(x, r) \mid x \in \O \land r > 0 \land B(x, r) \subseteq \O\right\} \subseteq \mathcal{P}(E).$$
    Soit $\U \subseteq B(\O)$, on définit une relation $\sim_{\U}$ (ou plus simplement $\sim$ s'il n'y a pas d'ambuiguité) sur $\U$ par :
    $$\forall u, v \in \U, \hspace{1em} u \sim_{\U} v \iff \exists w \in \U, u \cup v \subseteq w.$$
  \end{definition}

  \begin{proposition}
    Pour tout $\U \subseteq B(\O)$, la relation $\sim_{\U}$ est une relation d'équivalence.
  \end{proposition}

  \begin{proof}
    Soit $u_1, u_2, u_3 \in \U$ trois boules.

    Pour la reflexivité, on considère $w = u_1 \in \U$ et on a bien $u_1 \sim u_1$.

    Pour la symétrie, la définition de la relation est clairement symétrique.

    Enfin pour la transitivité, on suppose que $u_1 \sim u_2$ et $u_2 \sim u_3$.
    On a alors $v, w \in \U$ tels que
    $$u_1 \subseteq v, u_2 \subseteq v, u_2 \subseteq w \text{ et } u_3 \subseteq w.$$
    Ainsi, $\emptyset \not= u_2 \subseteq v \cap w$, or $v$ et $w$ sont des boules d'un espace ultramétrique. On a alors deux cas :
    \begin{itemize}
      \item $v \subseteq w$ dans ce cas on a $u_1 \stackrel{v}{\sim} u_3$
      \item $w \subseteq v$ et alors on a $u_1 \stackrel{w}{\sim} u_3$
    \end{itemize}
    \begin{figure}[ht]
      \centering
	\begin{tikzpicture}
      [
	level 1/.style={sibling distance=15mm},
	level 2/.style={sibling distance=15mm},
      ]
	\node {$v$}
		child {node {$u_1$}}
		child {
		    node {$w$}
		    child {node {$u_2$}}
		    child {node {$u_3$}}
		};
	\end{tikzpicture}
	\hspace{1.5cm}
	\begin{tikzpicture}
      [
	level 1/.style={sibling distance=15mm},
	level 2/.style={sibling distance=15mm},
      ]
	\node {$w$}
		child {
		    node {$v$}
		    child {node {$u_1$}}
		    child {node {$u_2$}}
		}
		child {node {$u_3$}};
	\end{tikzpicture}

  \end{figure}
  \pagebreak
  Donc $\sim_{\U}$ est bien une relation d'équivalence sur $\U$.
  \end{proof}

  Pour la suite de la section on fixe $\U \subseteq B(\O)$ un sous-ensemble de l'ensemble des boules incluses dans $\O$.

  \begin{definition}
    Soit $u \in \U$, la classe d'équivalence de $u$ pour la relation $\sim$ est noté $\bar{u}$.
    On introduit également $\B_u \in B(\O)$ l'union des éléments de la classe d'équivalence de $u$ :
    $$\B_u := \bigcup_{u \sim u'}{u'} \subseteq \O.$$
    Remarquons que $\B_u$ est une boule dès lors que cette union est bornée (d'après le lemme \ref{lem:union_mem_balls}),
    or cette $\B_u \subseteq \O$ et $\O$ est borné donc $\B_u$ est toujours une boule.
  \end{definition}

  \begin{definition}
    Soit $R$ un système complet de représentant pour la relation $\sim$.
    On pose alors
    $$\U' := \U \cup \left\{\B_u \mid u \in R\right\} \subseteq B(\O).$$
    car $\B_u \in B(\O)$ pour chaque $u \in R$.
  \end{definition}

  On considère désormais $\sim' = \sim_{\U'}$ qui est encore une relation d'équivalence,
  et soit $R'$ un système complet de représentant pour la relation $\sim'$.

  \begin{theorem}
    Soit $\O$ un ouvert borné de $E$. Avec les notations introduites précédemment on a
    $$\O = \bigsqcup_{u\in R'}{\B_u}.$$
  \end{theorem}

  \begin{proof}
    On commence par montrer que cette union est bien égale à $\O$.
    En considérant $\U = \left\{B(x, r_x) \mid x \in \O\right\}$,
    \begin{align*}
      \O = \bigcup_{u \in \U}{u} = \bigcup_{u \in \U'}{u} = \bigcup_{u\in R'}{\left(\bigcup_{v \sim' u}{v}\right)}
				= \bigcup_{u\in R'}{\B_u}
    \end{align*}

    Il reste encore à montrer que cette union est disjointe.
    Supposons que $\B_u \cap \B_v \not= \emptyset$.
    Alors on a $\B_u \subseteq \B_v$ ou $\B_v \subseteq \B_u$ par le lemme \ref{lem:ball_subset_trichotomy}.
    Dans les deux cas on a $u \sim' v$ avec $w = \B_v \in \U'$ dans le premier cas et $w = \B_u \in \U'$ dans le second.
    Or $R$ est un système complet de représentant pour $\sim'$ et donc nécessairement $u = v$.
    Finalement les $(\B_u)_{u\in R'}$ sont deux à deux disjoints et leur union est égale à $\O$,
    ce qui conclut la preuve du théorème \ref{thm:open_bdd_eq_disjoint_union_ball}.
  \end{proof}

  % Avec la même preuve on obtient en changeant une hypothèse la même conclusion.
  %
  % \begin{corollary}
  %   Soit $E$ un espace ultramétrique tel qu'il existe $(x_i)_{i\in I} \in E^I$ et $(r_i)_{i\in I}$ des réels strictement positifs vérifiant
  %   la propriété suivante :
  %   $$E = \bigcup_{i\in I}{B(x_i, r_i)} \hspace{1cm} \text{et} \hspace{1cm} \sup_{i\in I}{r_i} < \infty.$$
  %   Alors pour tout ouvert $\O$ de $E$, il existe $R \subseteq E$ et $r \colon R \longrightarrow \R^*_+$ tel que
  %   $$\O = \bigsqcup_{x\in R}{B(x, r(x))}.$$
  % \end{corollary}
  %
  % \begin{proof}
  %   On réécrit $\O$ comme l'union suivante :
  %   $$\O = \bigsqcup_{i\in I}{B(x_i, r_i) \cap \O}.$$
  %   Cette union est disjointe car les boules des espaces ultramétriques sont disjointe ou bien incluses l'une dans l'autre,
  %   donc on peut se ramener à ce cas (\textsc{A vérifier}). % TODO
  %   Ainsi, pour tout $i \in I$, $B(x_i, r_i) \cap \O$ est un ouvert borné et donc peut être écrit comme une union disjointe de boule.
  %   De cette manière on obtient le résultat voulu.
  % \end{proof}
\end{document}
