\documentclass[../../rapport.tex]{subfiles}

\begin{document}
  Pour prouver une proposition dans la théorie des types, on utilise des règles d'inférences, ou plus simplement règles, de la forme
  $$
  \prfinterspace=2em
  \prftree[r]{\textsc{Nom}}{\mathcal I_1}{\hdots}{\mathcal I_k}{\mathcal I}
  $$
  où $\mathcal I$ est la conclusion de la règle tandis que les $\mathcal I_1, \hdots \mathcal I_k$
  sont les prémisses ou hypothèses de la règle.

  Avec de telles règles, on peut prouver une proposition en construisant un arbre de preuve
  dont la racine est la proposition à prouver et les liens entre les noeuds respectent les règles d'inférence de la théorie.

  \subsubsection{Contexte}

  Les règles concernant le jugement de contexte sont au nombre de deux et permettent de vérifier qu'un contexte $\Gamma$ est bien formé,
  c'est-à-dire que toutes les variables et jugements apparaissant dans $\Gamma$ sont dans un ordre cohérent.
  Pour ce faire on a ces règles :

  $$
  \begin{array}{cc}
    \prfbyaxiom{($\emptyset\ctx$)}{\cdot\ \ctx} \hspace{1.5cm}
    & \prftree[r]{($\ctx$)}{a_1 : A_1, \hdots, a_{n-1} : A_{n-1} \vdash A_n\ \text{type}}
      {(a_1 : A_1, \hdots, a_n : A_n) \ctx}
  \end{array}
  $$
  Le jugement "$A_n\ \text{type}$" signifie que $A_n$ est un type et sera détaillé davantage dans la partie suivante.
  La première règle dit qu'un contexte vide est toujours bien formé, la seconde règle dit que un contexte est bien formé
  dès lors que la dernière variable est bien typé sous l'hypothèse que le contexte auquel on retire la dernière variable
  est lui-même bien formé.

  \subsubsection{Égalité}

  L'égalité par définition "$\equiv$" se comporte comme l'égalité classique, pour se faire elle vérifie les mêmes propriétés élémentaires,
  à savoir la reflexivité, la symétrie et la transitivité que l'on exprime avec les règles suivantes.

  $$
  \prfinterspace=2em
  \begin{array}{ccc}
    \prftree[r]{(refl)}{\Gamma \vdash a : A}{\Gamma \vdash a \equiv a : A} \hspace{0.5cm}
    & \prftree[r]{(sym)} {\Gamma \vdash b \equiv a : A}{\Gamma \vdash a \equiv b : A} \hspace{0.5cm}
    & \prftree[r]{(trans)}{\Gamma \vdash a \equiv b : A}{\Gamma \vdash b \equiv c : A}{\Gamma \vdash a \equiv c : A}
  \end{array}
  $$

  On ajoute aussi deux règles supplémentaires qui permettent d'échanger le rôle de deux élément ou type sous l'hypothsèse qu'ils sont égaux.

  $$
  \prfinterspace=2em
  \begin{array}{cc}
    \prftree[r]{($\equiv$ type)}{\Gamma \vdash a : A}{\Gamma \vdash A \equiv B}{\Gamma \vdash a : B} \hspace{1cm}
    & \prftree[r]{($\equiv$ eq)}{\Gamma \vdash a \equiv b : A}{\Gamma \vdash A \equiv B}{\Gamma \vdash a \equiv b : B}
  \end{array}
  $$

  Enfin, on ajoute une dernière règle permettant d'utiliser des jugements présent dans le contexte, qui s'exprime sous la forme suivante.

  $$
  \prftree[r]{(var)}
    {(a_1 : A_1, \hdots, a_n : A_n) \ctx}
    {a_1 : A_1, \hdots, a_n : A_n \vdash a_i : A_i}
  $$

\end{document}
