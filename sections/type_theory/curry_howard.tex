\documentclass[../../rapport.tex]{subfiles}

\begin{document}
  Dans la théorie des types, les propositions sont donc des types et les preuves de ces propositions sont donc des éléments de ce type.
  Ainsi, prouver un théorème correspond à construire un élément (ou un terme) du type correspondant à l'énoncé du théorème.
  On a décrit comment construire des types de manière générale et ces constructions peuvent être utilisés pour construire des propositions,
  comme décrit dans le tableau suivant.

  \begin{figure}[ht]
    \centering
    \begin{tabular}{m{3cm} l}
      % \hline
      Logique & Type \vspace{0.4em} \\
      \hline
      \vspace{0.4em}
      $A \implies B$ & $A \fun B$ \\
      % $\forall x : A, P(x)$ & $\prod_{x:A}P(x)$ \\
      $A \wedge B$ & $A \times B$ \\
      % $\exists x : A, P(x)$ & $\Sigma_{x:A}P(x)$ \\
      $\bot$ & $\0$ \\
      $\top$ & $\1$ \\
      $A \vee B$ & $A + B$ \\
      $\neg A$ & $A \fun \0$ \\
      $\forall x : A, P(x)$ & $\prod_{x:A}P(x)$ \\
      $\exists x : A, P(x)$ & $\Sigma_{x:A}P(x)$ \\
      % \hline
    \end{tabular}
  \end{figure}

  On peut donc voir les règles d'inférences comme des règles de construction des preuves dans la théorie des types.
  Par exemple, pour prouver une implication $A \implies B$, en théorie des types il faut construire une fonction $f : A \fun B$.
  La fonction $f$ peut alors être vu comme une fonction prenant des preuves de $A$ et retournant une preuve de $B$
\end{document}
