\documentclass[../../rapport.tex]{subfiles}

\begin{document}
  Cette section est un retour d'expérience sur l'utilisation de Lean en tant qu'outil de formalisation pour les
  mathématiques.

  Durant ce TER, j'ai donc utilisé Lean pour formaliser le théorème de Bowen, dont un preuve figure dans mon rapport de stage
  en annexe (preuve initialement dûe à R. Bowen).
  Ce théorème établit l'existence et l'unicité d'une certaine mesure de probabilité dite de Gibbs sur un espace métrique particulier,
  et donc ce théorème fait intervenir notamment de la topologie et de la théorie de la mesure.

  Pour formaliser ce théorème, j'ai donc eu recours à des outils extérieurs à Lean comme
  \href{https://github.com/PatrickMassot/leanblueprint}{leanblueprint},
  qui permet à partir d'un texte mathématiques d'obtenir un graphe décrivant comment le résultat final dépend
  des lemmes, théorèmes et définitions intermédiaires et d'en donner l'état d'avancement dans le code Lean associé.

\end{document}
